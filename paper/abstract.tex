\section*{Abstract}

The need for accurate energy measurement tools for computing systems today
grows with the increasing demand for computational power. The rapid
development of software systems, directly or indirectly leveraging
\gls{AI}, and critical infrastructure applications which require the use of
\gls{HPC} (such as flood or tsunami simulations) are the main drivers of this
growth. In addition, global trends toward sustainability and responsible
energy consumption further emphasize the importance of high-quality energy
profiling.

As \gls{CPU} represents the core component of a modern computing system,
accurate measurement of its energy consumption plays a crucial role.
Such data can be applied in multiple contexts, including the software
solutions optimization, energy-aware billing models, and performance
evaluation in \gls{HPC} environments.

This paper will look at the two main players of today's market, Intel and AMD.
While Intel has long dominated the \gls{CPU} market, AMD has also developed
processors targeting server and cluster workloads, providing various telemetry
capabilities including support for Intel’s \gls{RAPL} interface and other
AMD-specific solutions. The latter include kernel modules and community-driven
open-source tools exposing telemetry features of AMD hardware.

This research aims to conduct a comparative analysis of energy measurement
interfaces available on AMD \gls{CPU}s, focusing on the evaluation of accuracy,
resolution, and usability. Specifically, the study contrasts Intel-inspired
\gls{RAPL} with AMD’s native telemetry mechanisms in order to determine their
strengths, limitations, and potential applications in modern energy-aware
computing.

\subsubsection*{Keywords}

Energy Measurements, CPU Power Consumption, RAPL, AMD Ryzen SMU,
Software Interfaces, Performance Monitoring, Linux, Kernel Drivers,
hardware topology, reproducibility, benchmarking, EMA.

\subsubsection*{Acknowledgements}

Athor wants to express his deep gratitude to the following people: Jessica
Ullmer-Hohstadt, Daniel Levoshich, Bohdan Sukhovarov, Johannes Spazier, Tobias
Jaeuthe, Cholatit Schomann, Alireza Mahmoud, Mazhar Hameed, Christian
Dombrowski, Mahmoudreza Babaei, Yuvraj Dhepe, Jahanvi Merchant, Taha Osmani,
Hiba Khalid, Saeed Zahedi, Mohammad Mahdavi, Dariya Afanasyeva and all staff
members of the GISMA Registry Team.

