\section{Introduction}

\subsection{Background}

It is widely acknowledged that the demand for computational power has
been steadily increasing in recent years, driven both by rapid development of
\gls{AI} and advances in \gls{HPC} applications. Recent studies show that
large-scale AI-models training can impose significant power (electrical)
requirements at the compute node level, highlighting the growing importance
of energy management in computational infrastructures \parencite{Latif_2025}.
Additionally, international assessments show that digital infrastructures
will continue to exert considerable pressure on global energy systems
\parencite{IEA2025_AI_Energy}.

This trend is further accentuated by the global emphasis on sustainable
development, where efficient utilization of energy and minimization of
environmental impact are increasingly prioritized. In \gls{HPC} environments
energy-aware practices have emerged as critical design criteria: evidenced by
initiatives such as \emph{Green500} ranking and related analytical studies
\parencite{GreenHPC}. In order to formulate effective optimization strategies,
regulatory guidelines and operational policies it is essential to obtain precise
measurements of energy consumption across various computational systems'
components, including \gls{CPU}, \gls{GPU}, memory, specific network-attached
hardware, and interconnects.

Several software frameworks and tools have been developed to facilitate
reliable collection of power and energy data (and also other sensor-type
metrics, out of scope of this work, e.g. temperature readings).

Representative examples include:
\begin{itemize}
  \item \emph{EMA} \footcite{EMA_git},
  \item \emph{Scaphandre} \footcite{Scaphandre},
  \item \emph{Alumet} \footcite{Alumet},
  \item \emph{EAR} \footcite{EAR}, and
  \item \emph{ClusterCockpit} \footcite{ClusterCockpit}.
\end{itemize}

Each of these tools aims to integrate energy measurement capabilities within
broader performance monitoring and management workflows while having various
subsets of features and/or limitations.

Among the components of modern computing systems, \gls{CPU}s occupy a central
role in both \gls{HPC} and \emph{Cloud} environments, so comprehension
of their energy characteristics is of great importance, particularly
in the context of evolving \gls{CPU} market. Although, Intel has historically
maintained a dominant position in server and desktop segments, AMD has
achieved notable gains over the past decade. Contemporary market analyses
indicate that AMD has captured a substantial share of both server and
desktop markets, with penetration exceeding 40\% among consumer systems
\parencite{NetworkWorld2025_Q1ServerShare}. Moreover, some economic researches
highlight the strategic significance of competitive dynamics
and innovation within the microprocessor industry
\parencite{GoettlerGordon2009_Competition}.

Energy measurement of \gls{CPU}s is typically facilitated through hardware- and
firmware-/software- level interfaces exposed to the \gls{OS} kernels.

Intel platforms predominantly employ the \gls{RAPL} mechanism, which provides
estimations of energy consumption for the \gls{CPU} package and its subdomains,
mitigating security risks (such as power-based side channel attacks
\parencite{PwrLeak_2023}) by including filtering behaviour for model-specific
registers.\parencite{IntelRAPL}

AMD processors offer a variety of telemetry mechanisms and \gls{SMU}s aside,
among which the \texttt{ryzen\_smu} \parencite{RyzenSMU_GitHub} driver provides
low-level access to power measurement readings. At the same time most of the
modern AMD \gls{CPU} model families also support \gls{RAPL} (with some slight
differences in realization).

Each interface presents distinct trade-offs in terms of measurements
accuracy, accessibility, and operational overhead. This motivates a
systematic comparative evaluation.

\subsection{Problem Statement}

The landscape of \gls{CPU} energy measurement interfaces on AMD platforms is
characterized by the coexistence of vendor-specific solutions and
adaptations of mechanisms originally designed for Intel \gls{CPU}s. In
particular, the availability of both the \texttt{ryzen\_smu} interfaces
and AMD's realization of the \gls{RAPL} mechanism introduces uncertainty in the
selection of appropriate tools for high-fidelity measurement and
integration with existing monitoring frameworks.

While \gls{RAPL} is widely adopted in academic and industrial contexts, its
port to AMD hardware differs in certain aspects and has its specific limitations.
Conversely, the \texttt{ryzen\_smu} driver offers access to low-level telemetry,
yet its practical benefits over \gls{RAPL} remain largely unexplored.

\subsection{Research Questions}

Considering the above following research questions were formulated:

\begin{enumerate}
  \item To what extent do energy measurements obtained from the
        \gls{SMU} (\gls{PM} Table) and \gls{RAPL} interfaces for the
        \emph{package} and \emph{core} power domains show similar
        dynamic and temporal trends?
      \item What is the magnitude of errors (\gls{RMSE}, \gls{MAE}, \gls{MAPE}),
        and to what degree are the two interfaces theoretically interchangeable
        in practical energy measurement tasks?
      \item How do differences in the supported sampling rates between
        \gls{RAPL} and \gls{SMU} affect the temporal resolution of measurements,
        and what are the implications for future high-frequency energy profiling?
\end{enumerate}

\subsubsection{Considerations}

Due to the time and resource limitations this study focuses primarily on the
energy measurements on \emph{Software Level}. That means that no objective
\emph{source of truth} (e.g. external \emph{Power Measurement Device}
attached to the \gls{PC}'s \gls{PSU}) was used.

Both interfaces research in this work provide power/energy estimations based
on some internal mechanisms based on combination of hardware counters and
mathematical estimation models. Which in turn means that the energy and/or
power values obtained through those mechanisms do not reflect the 100\%
accurate real/physical consumption.
