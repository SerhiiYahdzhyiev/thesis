\section{Test/Experiment System Setup Details}
\label{app:setup}

This appendix provides detailed information about the hardware and software
configuration of the test system used in this study, as well as relevant runtime
settings. The goal is to ensure reproducibility of the presented measurements.
The information is gathered using \texttt{lscpu}, \texttt{dmidecode},
\texttt{lsblk}, and \texttt{lspci}.

\subsection{Hardware Specifications}
\label{app:hwspec}

\begin{table}[h]
\caption{Hardware Components Details (CPU)}
\subsubsection*{CPU}
\begin{tabularx}{\linewidth}{>{\hsize=0.3\hsize}X >{\hsize=0.7\hsize}X}
\toprule
\textbf{Component} & \textbf{Details} \\
\midrule
Model & AMD Ryzen 5 3600 (6-Core Processor) \\
Architecture & x86\_64 (64-bit, little endian) \\
Cores / Threads & 6 cores / 12 threads \\
Base / Boost Frequency & 2.2 GHz (min) -- 4.2 GHz (max) \\
Cache & L1: 192 KiB, L2: 3 MiB, L3: 32 MiB \\
Virtualization & AMD-V \\
NUMA nodes & 1 (0--11) \\
\bottomrule
\end{tabularx}
\end{table}

\subsubsection*{Motherboard and BIOS}
\begin{table}[h]
\caption{Hardware Components Details (Motherboard and BIOS)}
\begin{tabularx}{\linewidth}{>{\hsize=0.3\hsize}X >{\hsize=0.7\hsize}X}
\toprule
\textbf{Component} & \textbf{Details} \\
\midrule
Manufacturer & Gigabyte Technology Co., Ltd. \\
Model & A320M-H-CF \\
Version & x.x \\
BIOS Vendor & American Megatrends Inc. \\
BIOS Version & F53 \\
BIOS Release Date & 01/05/2021 \\
\bottomrule
\end{tabularx}
\end{table}

\subsubsection*{Memory}
\begin{table}[h]
\caption{Hardware Components Details (Memory)}
\begin{tabularx}{\linewidth}{>{\hsize=0.3\hsize}X >{\hsize=0.7\hsize}X}
\toprule
\textbf{Component} & \textbf{Details} \\
\midrule
Total Installed & 32 GB DDR4 \\
Configuration & 2 $\times$ 16 GB \\
Speed & 2666 MT/s \\
Error Correction & None (non-ECC) \\
\bottomrule
\end{tabularx}
\end{table}

\subsubsection*{Storage Devices}
\begin{table}[h]
\caption{Hardware Components Details (Storage)}
\begin{tabularx}{\linewidth}{>{\hsize=0.3\hsize}X >{\hsize=0.7\hsize}X}
\toprule
\textbf{Device} & \textbf{Details} \\
\midrule
sda & Crucial MX500 SSD (CT500MX500SSD1) -- 465.8 GB \\
sdb & GOODRAM CX400 SSD (SSDPR-CX400-01T-G2) -- 953.9 GB \\
sdc & Samsung 870 EVO SSD -- 931.5 GB \\
sdd & Samsung 840 EVO SSD -- 111.8 GB \\
\bottomrule
\end{tabularx}
\end{table}

\subsection{Software Environment}
\label{app:swspec}

The experiments were conducted on an up-to-date Arch Linux system at the time
of testing. The following table summarizes the operating system, kernel,
drivers, and software environment used in the study.

\begin{table}[h]
\caption{Software Components Details}
\begin{tabularx}{\textwidth}{>{\hsize=0.3\hsize}X >{\hsize=0.7\hsize}X}
\toprule
\textbf{Component} & \textbf{Details} \\
\midrule
\multicolumn{2}{c}{\textbf{System}} \\
\midrule
Operating System & Arch Linux (rolling release) \\
Kernel & 6.16.4-arch1-1 \\
DKMS Modules & AMDPowerProfiler/10.5, ryzen\_smu/162.e61177d \\
\midrule
\multicolumn{2}{c}{\textbf{Toolchain and Packages}} \\
\midrule
GCC & 15.2.1 (default), 14.3.1 (compatibility) \\
Clang & 20.1.8 \\
Linux Headers & 6.16.4-arch1-1, 6.12.44-1-lts \\
Firmware & linux-firmware (20250808), including amdgpu, nvidia, radeon, etc. \\
Other relevant packages & util-linux 2.41.1 \\
\bottomrule
\end{tabularx}
\end{table}

\textbf{Systemd running services dump:}
\begin{verbatim}
UNIT                            LOAD   ACTIVE SUB     DESCRIPTION
init.scope                      loaded active running System and Service Manager
session-11.scope                loaded active running Session 11 of User root
dbus-broker.service             loaded active running D-Bus System Message Bus
getty@tty3.service              loaded active running Getty on tty3
iwd.service                     loaded active running Wireless service
systemd-journald.service        loaded active running Journal Service
systemd-logind.service          loaded active running User Login Management
systemd-networkd.service        loaded active running Network Configuration
systemd-resolved.service        loaded active running Network Name Resolution
systemd-timesyncd.service       loaded active running Network Time Synchronization
systemd-udevd.service           loaded active running Rule-based Manager for
    	    	    	    	Device Events and Files
systemd-userdbd.service         loaded active running User Database Manager
user@0.service                  loaded active running User Manager for UID 0
dbus.socket                     loaded active running D-Bus System Message Bus Socket
systemd-journald-dev-log.socket loaded active running Journal Socket (/dev/log)
systemd-journald.socket         loaded active running Journal Sockets
systemd-networkd.socket         loaded active running Network Service Netlink Socket
systemd-udevd-control.socket    loaded active running udev Control Socket
systemd-udevd-kernel.socket     loaded active running udev Kernel Socket
systemd-userdbd.socket          loaded active running User Database Manager Socket

Legend: LOAD   → Reflects whether the unit definition was properly loaded.
    ACTIVE → The high-level unit activation state, i.e. generalization of SUB.
    SUB    → The low-level unit activation state, values depend on unit type.

20 loaded units listed.
\end{verbatim}

